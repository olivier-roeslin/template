\section{Manipulation 4 "Déterminer expérimentalement l'évolution du module
de Young en fonction de la longueur des barreaux pour au moins 
un matériau."}
\subsection{Approches / Méthodes}
\subsubsection{\large Rappels Théoriques}
\paragraph{Introduction}
Le Module de Young, la vitesse de propagation du son dans un solide et la densité étant des 
caractéristiques intrinsèques aux matériaux, si on varie la longueur de notre barreau, le module de 
Young devrait rester constant. Le but de cette expérimentation est de confirmer ce postulat en 
mesurant le module de Young de trois barreaux de longueurs différentes et de mêmes
matériaux et comparer les valeurs trouvées.
%\subsubsection{Les Rappels Théoriques}

%\paragraph{Liste des paramètres}
%\begin{itemize}
    %\item $\rho:$ masse volumique du matériel $[\frac{kg}{m^3}]$
    %subitem $m_{1}:$ masse $[kg]$
    %\subitem $m_{2}:$ masse immergée $[kg]$
    %\subitem $\rho_{l}:$ masse volumique du liquide $[\frac{kg}{m^3}]$
    %\item $E:$ module le Young du matériel $[Pa]$
    %\item $v:$ vitesse de propagation de l'onde $[\frac{m}{s}]$
%\end{itemize}
\subsubsection{\large Protocole expérimental}
Le protocole est le même que pour la manipulation précédente, à la nuance près que
l'essai se fait sur trois barreaux de même matériel et de longueur variable(100-200-400 mm).
Le matériel sera de l'acier inox, parce que c'est le seul dont nous disposons en 3 exemplaires de tailles 
différentes.

\paragraph{Présentation du montage}
Les montages utilisés pour la mesure de la vitesse de propagation du son et celui pour la masse volumique 
sont exactement les mêmes que dans les expérimentations précédentes. 

\paragraph{Liste du matériel}
Le matériel utilisé est le même qu'à la manipulation 4, à l'exception près des barreaux.
\begin{itemize}   
    \item Tige matériaux et longueurs avec un diamètre de 10 mm :
    \subitem Acier inox 100 mm
    \subitem Acier inox 200 mm
    \subitem Acier inox 400 mm
\end{itemize}
\paragraph{Méthode de mesure}
La méthode de mesure et celle de calcul des incertitudes sont exactement les mêmes que dans 
l'expérimentation précédente. Affin de comparer les résultats en fonction de la longueur du barreau, 
un graphique à moustaches sera effectué.

\newpage

\subsection{\large Résultats / Analyse}
\subsubsection{\large Données brutes}
\paragraph{\large Cylindre Aluminium 100 mm}
\paragraph{Tableau}
\paragraph{Graphe}
\paragraph{Observations}

\paragraph{\large Cylindre Acier inox 200 mm}
\paragraph{Tableau}
\paragraph{Graphe}
\paragraph{Observations}

\paragraph{\large Cylindre Acier inox 400 mm}
\paragraph{Tableau}
\paragraph{Graphe}
\paragraph{Observations}

\newpage

\subsubsection{\large Données réduites}
\paragraph{\large Cylindre Acier inox 100 mm}
\paragraph{Calculs}
\paragraph{Incertitudes}

\paragraph{\large Cylindre Acier inox 200 mm}
\paragraph{Calculs}
\paragraph{Incertitudes}

\paragraph{\large Cylindre Acier inox 400 mm}
\paragraph{Calculs}
\paragraph{Incertitudes}

\newpage

\subsubsection{\large Tableau récapitulatif}
\paragraph{Valeurs}
\paragraph{Incertitudes relatives}
\paragraph{Ecart relatif}

\subsubsection{\large Discussion quantitative}
\paragraph{Incertitude > écart}
validation
\paragraph{Incertitude < écart}
invalidation
discussion critique des méthodes