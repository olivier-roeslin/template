\section{Manipulation 2 "La vitesse de propagation du son dans des
barreaux solides d'au moins 3 matériaux différents."}
\subsection{Approches / Méthodes}
\subsubsection{\large Rappels Théoriques}
\paragraph{Introduction}
Quand une onde traverse un solide, 
sa célérité est tout sauf uniforme.
La célérité à laquelle une onde se déplace
dans un solide dépend intrinsèquement 
des caractéristiques de ce dernier.
Prenons l'exemple d'un violoncelle, 
le bois de l'instrument est choisi 
en raison de ses propriétés acoustiques 
particulières, notamment sa densité, son 
élasticité et sa configuration interne, 
favorisant ainsi une résonance optimale 
et une amplification efficace des vibrations 
des cordes.
Cette mélodie est rendue possible par la 
spécificité du bois du violoncelle, qui 
offre une célérité idéale pour la transmission 
des ondes sonores.
Maintenant, si nous prenons l'exemple d'une 
mine souterraine, où des explosifs sont 
utilisés pour creuser des tunnels.
Le son de l'explosion se propage à travers 
la roche, mais à une célérité bien plus 
élevée que dans le bois du violoncelle. 
Cette augmentation de vitesse est due à la 
nature plus rigide et dense de la roche, 
ce qui permet aux ondes sonores de se 
déplacer plus rapidement. Le but de cette 
expérimentation est de confirmer cela 
en mesurant la célérité du son dans différents 
matériaux.
\subsubsection{\large Protocole expérimental}
Le protocole expérimental est identique 
à celui de l'expérience précédente (Manipulation 1), 
à l'exception du fait que l'expérimentation 
est réalisée avec des barres de différents 
matériaux et de longueurs variables 
(100-200-400 mm).
\paragraph{Présentation du montage}
La configuration expérimentale employée 
pour mesurer la vitesse de propagation 
de l'onde est identique à l'expérience 
antérieure (Manipulation 1).

\newpage

\paragraph{Liste du matériel}
Le matériel utilisé est le même que 
celui de la première manipulation, 
avec l'ajout des barreaux suivants pour 
effectuer les mesures :
\begin{itemize}
    \item Tige matériaux et longueurs avec un diamètre de 10 mm:
    \subitem Aluminium 100, 200 mm.
    \subitem Laiton 100 mm.
    \subitem Cuivre 100 mm.
    \subitem Bois dur 200 mm.
    \subitem PVC 200 mm.
    \subitem Verre 200 mm.
    \subitem Verre acrylique 200 mm.
    \subitem Pom 200 mm.
    \subitem Nylon 200 mm.
\end{itemize}
\paragraph{Méthode de mesure}
La méthode de mesure et de calcul des 
incertitudes reste inchangée par rapport 
à l'expérience précédente (Manipulation 1). 
Afin de comparer les résultats en fonction 
du matériau, un graphique sera réalisé. 
Nous réaliserons les mesures de vitesses de 
propagation du son dans des barreaux solides 
d'au moins 3 matériaux différents.

\newpage

\subsection{\large Résultats / Analyse}
\subsubsection{\large Données brutes}
\paragraph{\large Aluminium 100 mm}
\paragraph{Tableau}
\paragraph{Graphe}
\paragraph{Observations}

\paragraph{\large Cylindre Acier inox 200 mm}
\paragraph{Tableau}
\paragraph{Graphe}
\paragraph{Observations}

\paragraph{\large Cylindre Acier inox 400 mm}
\paragraph{Tableau}
\paragraph{Graphe}
\paragraph{Observations}

\newpage

\subsubsection{\large Données réduites}
\paragraph{\large Cylindre Acier inox 100 mm}
\paragraph{Calculs}
\paragraph{Incertitudes}

\paragraph{\large Cylindre Acier inox 200 mm}
\paragraph{Calculs}
\paragraph{Incertitudes}

\paragraph{\large Cylindre Acier inox 400 mm}
\paragraph{Calculs}
\paragraph{Incertitudes}

\newpage

\subsubsection{\large Tableau récapitulatif}
\paragraph{Valeurs}
\paragraph{Incertitudes relatives}
\paragraph{Ecart relatif}

\subsubsection{\large Discussion quantitative}
\paragraph{Incertitude > écart}
validation
\paragraph{Incertitude < écart}
invalidation
discussion critique des méthodes