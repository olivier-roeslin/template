\section{Introduction}
\subsection{Contexte}

\subsubsection{\large Le cadre}
Le présent projet de laboratoire est réalisé dans le cadre de notre 
formation de bachelor d'ingénieur microtechnicien dans le 
module Phy3. Il se déroulera tout au long du semestre d'automne 
2023, à une fréquence de quatre périodes toutes les deux semaines. Nous allons 
le réaliser à trois avec Rivera Evan, Roeslin Olivier et 
Pitteloud Célien. Nous sommes encadrés par M. Gravier, maître d'enseignement
et Mme Hanin, assistante HES.\\ 
Le but de ce projet est l'étude des ondes acoustiques dans la matière solide.
Ce travail nous permet, en plus de nous familiariser avec les exigences de rigueur 
lors des expérimentations et de la rédaction d'un rapport de laboratoire, de faire 
le lien avec des notions vues dans différents cours. En effet, les phénomènes
ondulatoires sont au programme du module de Phy3 dans lequel a lieu ce laboratoire.
Le module de Young est une notion étudiée et utilisée dans les cours de ResMatMT
et Matx1MT.

\subsubsection{\large Etat de l'art (SOTA)}
Une onde est la propagation d'une perturbation produisant sur son
passage une variation réversible des propriétés physiques locales 
du milieu. Elle se déplace avec une vitesse déterminée qui dépend 
des caractéristiques du milieu de propagation.
Il existe trois principaux types d'ondes: les ondes mécaniques, 
les ondes électromagnétiques et les ondes gravitationnelles.
Ces trois types varient en longueur d'onde et comprennent, 
pour les ondes mécaniques, les infrasons, 
les sons et les ultrasons.~\cite{wikipedia-onde}

%\footnote{WIKIPEDIA. 
%\textit{Onde --- Wikipédia, l'encyclopédie libre}. [en ligne]. 2023. 
%[consulté le 04.10.2023]. Disponible sur : 
%\href{https://fr.wikipedia.org/wiki/Onde}{wikipedia.org}}\\

Dans ce rapport, nous allons nous concentrer sur l'étude de la propagation
des ondes acoustiques dans les solides, autrement dit les ondes 
mécaniques progressives. 

Ce phénomène physique repose sur l'oscillation des particules 
d'un milieu et de la transmission de ces oscillations.
Ces dernières peuvent avoir plusieurs directions par rapport à la 
direction de propagation d'une onde.
Par exemple, si l'on jette un caillou dans un étang
et que l'on regarde les ondes créées autour du point d'impact,
l'oscillation est dite transversale. Elle est perpendiculaire
à la direction de propagation de l'onde, c'est un mouvement de cisaillement.

Par opposition, si nous prenons un ressort avec de nombreuses 
spires auxquelles nous agitons l'extrémité et que l'on regarde le 
déplacement et l'oscillation des spires, la propagation de l'onde sera
dans la même direction que l'oscillation des particules.
On dira que ce type d'onde est une onde longitudinale, une suite de
compressions et de dilatations.~\cite{royer-dieulesaint-acoustique}

%\footnote{DANIEL ROYER, EUGENE DIEULESAINT. \textit{Acoustique - 
%Propagation dans un solide --- Techniques de l'ingénieur}. 
%[en ligne]. 2023. [consulté le 14.10.2023]. 
%Disponible sur : \href{https://www.techniques-ingenieur.fr/base-documentaire/sciences-fondamentales-th8/applications-en-mecanique-physique-42643210/acoustique-af3814/}
%{techniques-ingenieur.fr}}

Chaque molécule qui constitue le ressort va osciller autour
d'une position d'équilibre, car il n'y a pas de transport de matière.
Cette perturbation va se propager le long du ressort.
Cette alternance de zone comprimée et de zone dilatée, va se retrouver dans 
les ondes sonores.~\cite{revisions-bac}

%\footnote{REVISIONS BAC. 
%\textit{Ondes transversales et longitudinales --- Youtube}. 
%[en ligne]. 2023. [consulté le 14.10.2023]. Disponible sur : 
%\href{https://www.youtube.com/watch?v=l1V4_SPfhLM}
%{youtube.com/@Revisionsbac}}\\

\newpage

En résumé, le principe sous-jacent à la propagation 
d'une onde dans un solide repose sur trois principes physiques fondamentaux:
\begin{enumerate}
    \item La compression et la dilatation:
    \subitem Lorsqu'une onde se propage dans un solide, 
    les particules du matériau subissent une alternance de 
    compression (rapprochement des particules) 
    et de dilatation (écartement des particules). 
    Cette variation périodique crée une série de régions plus denses 
    (compression) et moins denses (dilatation) à mesure que l'onde se 
    propage.
    \item Le transfert d'énergie:
    \subitem  Les particules du matériau ne se déplacent pas de manière 
    significative, mais elles transmettent leur énergie cinétique les unes 
    aux autres. C'est ce transfert d'énergie qui permet à l'onde de 
    se propager à travers le solide. Les particules vibrent autour de leur 
    position d'équilibre, transmettant l'énergie de proche en proche.
    \item La propagation ondulatoire:
    \subitem Les perturbations de compression et de dilatation se propagent 
    sous forme d'ondes mécaniques à travers le matériau. Ces ondes peuvent 
    être longitudinales ou transversales. La vitesse de 
    propagation dépend des propriétés du matériau, telles que sa densité, 
    sa rigidité et sa cohésion.\\
\end{enumerate}

Le module de Young ou module d'élasticité est une caractéristique propre à
chaque matériau. Ce module définit un rapport entre une contrainte et la
déformation d'un matériau. Cette caractéristique modifie la vitesse de
propagation des ondes acoustiques dans le solide.
Tout solide possède une fréquence naturelle qui dépend de l'élasticité de
l'objet, et donc de son module de Young, à laquelle il vibre avec la plus
grande amplitude possible. Lorsqu'un objet est exposé à des vibrations de
sa fréquence naturelle, il se met à vibrer aussi. Ce phénomène s'appelle
la résonance.~\cite{professeur-khattabi}\\

%\footnote{PROFESSEUR KHATTABI. 
%\textit{Résonance --- DSFM}. 
%[en ligne]. 2023. [consulté le 14.10.2023]. Disponible sur : 
%\href{https://web.dsfm.mb.ca/ecoles/clr/profs/fkhattabi/M3L2.pdf}
%{web.dsfm.mb.ca}}\\

De plus, lorsqu'un matériau est exposé à une variation de température, son
module de Young va varier en raison de la modification de la nature des
liaisons interatomiques assurant la cohésion du matériau.

\newpage

\subsubsection{\large Domaines d'applications}
L'étude de la propagation d'ondes est utilisée dans beaucoup de secteurs 
pour inspecter ce que l'on ne peut pas voir à l'intérieur d'un matériau
solide. Voici quelques exemples d'utilisation de ce principe physique:

- Secteur médical:
les ondes électromagnétiques sont beaucoup utilisées pour l'imagerie médicale. 
Elles sont envoyées dans le corps des patients et leur comportement est modifié
en fonction du solide qu'elles traversent, ce qui permet d'obtenir des images
de l'intérieur du corps des patients. Ces ondes peuvent avoir des fréquences et
longueur d'ondes différentes pour des utilisations différentes (les rayons X
pour les radiographies, les micro-ondes pour les échographies, les ondes radio
pour les IRM).~\cite{cea}\\

%\footnote{CEA. 
%\textit{Le fonctionnement d'un scanner X --- CEA, De la recherche à l'industrie}. [en ligne]. 2014. 
%[consulté le 15.10.2023]. Disponible sur : 
%\href{https://www.cea.fr/multimedia/pages/videos/culture-scientifique/sante-sciences-du-vivant/fonctionnement-scanner-x.aspx}{cea.fr}}\\

- Secteur de la sismologie:
à la suite d'un séisme, des vibrations sont transmises à la terre. Celles-ci se
déplacent à des vitesses différentes en fonction du type de matières qu'elles
traversent. Ces ondes sont mesurées à l'aide d'un sismographe. Ces mesures
permettent de comprendre la structure du sol et de plus ou moins prévoir les
futurs séismes.~\cite{wikipedia-echographie}\\

%\footnote{WIKIPEDIA. 
%\textit{Échographie --- Wikipédia, l'encyclopédie libre}. [en ligne]. 2023. 
%[consulté le 15.10.2023]. Disponible sur : 
%\href{https://fr.wikipedia.org/wiki/Échographie}{wikipedia.org}}\\

- Secteur industriel:
la technologie d'imagerie par échographie est aussi utilisée en industrie. En
effet, en étant envoyées à travers une soudure ou une tôle de métal, les
ultrasons vont se déplacer différemment dans la matière, si le solide est de
bonne ou mauvaise qualité, par exemple.\\

- Secteur minier:
à la manière de la sismologie, les ondes sismiques peuvent être utilisées pour
détecter des gisements minéraux, comprendre la structure géologique du sol.
Des ondes peuvent être émises dans le sol par des activités humaines (camion
vibreur, explosif, canon à air, etc.).~\cite{wikipedia-imagerie-sismique}

%\footnote{WIKIPEDIA. 
%\textit{Imagerie sismique --- Wikipédia, l'encyclopédie libre}. [en ligne]. 2023. 
%[consulté le 15.10.2023]. Disponible sur : 
%\href{https://fr.wikipedia.org/wiki/Imagerie_sismique}{Wikipedia.org}}

\newpage

\subsection{Objectifs}
Au travers de nos différents objectifs, le but de ce projet est de valider
expérimentalement les modules de Young de différents matériaux et de mesurer
les vitesses de propagation du son dans des solides en variant leurs
géométries et leurs températures.\\
\begin{enumerate}
    \item Concevoir et réaliser un montage expérimental permettant 
    de mesurer par méthode de résonance en mode continu la vitesse 
    de propagation du son dans des barreaux solides.
    \item Mesurer la vitesse de propagation du son dans des barreaux 
    solides d'au moins 3 matériaux différents.
    \item Déterminer expérimentalement le module de Young d'au moins
    3 matériaux.
    \item Déterminer expérimentalement l'évolution du module de Young 
    en fonction de la longueur des barreaux pour au moins un matériau.
    \item Déterminer expérimentalement l'évolution du module de Young 
    en fonction de la température pour au moins un matériau.
\end{enumerate}

\subsubsection{\large Objectifs spécifiques et listes de tâches}
Tous les objectifs listés ci-dessus ont été fixés suivant la 
méthode SMART (Spécifiques, Mesurables, Atteignables, Réalistes 
et Temporels) en accord avec M. Gravier. Ils seront effectués 
dans l'ordre dans lequel ils sont listés et le numéro 5 est un 
objectif complémentaire que l'on effectuera si nous prenons 
suffisamment d'avance sur les objectifs 1 à 4. \\
Les tâches s'organiseront comme suit:\\
\begin{enumerate}
	\item Conception de notre montage expérimental
	\item Commande du matériel manquant
	\item Réalisation et vérification de notre montage\\
\end{enumerate}
Pour chacun des objectifs suivants, la séquence sera la même, soit:\\
\begin{enumerate}
	\item Établissement des lois et équations physiques
	\item Réalisation des mesures avec leur incertitude
	\item Réduction et analyse des résultats
	\item Rédaction du chapitre lié à l'objectif
\end{enumerate}